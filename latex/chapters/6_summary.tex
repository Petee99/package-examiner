\Chapter{Összefoglalás}

A szakdolgozat elkészítése magában hordozta az npm mélyrehatóbb megismerését és egy olyan szoftver megtervezését, implementálását, amely ezeket az ismereteket fel tudja használni a működésének során.\\

A Node.JS használata összesen egy tárgyhoz kapcsolódott, illetve akkor is kimerült az "npm install" és "npm run" parancsok használatával, széleskörű ismeretekkel nem rendelkeztem ezen technológiával kapcsolatban.\\

A dolgozat elkészítése során megismertem az npm és az npm registry működését, megterveztem és implementáltam egy olyan webes applikációt, amely képes kommunikálni a registry-vel és az adatokat felhasználva megfelelő lépéssorozatok eredményeként a szükséges információkat prezentálja. Elkészült egy olyan egyéni algoritmus, mely biztosítja a függőségi fa helyes megjelenítését.\\

Az interaktivitás megtartása érdekében le kellett küzdeni a lokális fájlok és processzek elérésével kapcsolatos kliens oldali JavaScript korlátozásokat. Ennek következtében ismeretem meg a GitHub Api-ban rejlő potenciált, amellyel a nyílt repositoryk vizsgálata a megszokott http kérésekkel megtehető.\\

A projekt elkészítése közben kiderült, hogy akár más megközelítéssel is megoldható ugyanez a problémakör, illetve a jelenlegi megoldás is kiegészíthető sok extra funkcióval. Egyelőre a redundáns funkcionalitást csak a kulcsszavak vizsgálatával ellenőrzi a program, ám ez a jövőben sokkal mélyebben és széleskörűbben megtehető a fájlok forráskódjának elemzésével. Erre azért lehet szükség, mivel jelenleg inkább ad javaslatot a felülvizsgálatra, mint elvégzi azt.\\

A szoftver fejlesztése során jobban megismertem, hogy aszinkron kérésekkel miként lehet dolgozni a JavaScriptben, hogy ennek milyen előnyei vannak és milyen nehézségekkel jár együtt. Fontos momentum volt még a szemantikus verziózás megismerése, mivel a verziók változásával akár a teljes függőségi fa is változhat.\\

A dolgozat az első ekkora hangvételű projektem, a témában részletezett feladatok elméleti hátterére és implementálására sor került, azonban azoknak a széleskörűbb használatot biztosító továbbfejlesztésére bőven akadnak még lehetőségek.

 



