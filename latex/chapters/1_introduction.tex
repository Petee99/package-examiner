\Chapter{Bevezetés}

A szoftverfejlesztés egy folyamatosan változó és fejlődő terület. Az utóbbi időben rengeteg különböző célú és felhasználású technológia jött létre, azonban akadnak olyanok is, amelyek ugyanazt a célt valósítják meg, más megközelítéssel.

Kezdetben leginkább a személyi számítógépekre fejlesztettek programokat. A célhardverek halmaza azonban ma már sokkal nagyobb és színesebb. Az okos eszközök, televíziók, háztartási gépek, számítógépek, és egyéb eszközök gyakran más-más operációs rendszereket futtatnak, amelyekre más-más technológiákkal lehet fejleszteni alkalmazásokat.

Mivel a céleszközök nagyobb része támogatja valamely, vagy akár több böngésző használatát, így célszerű lehet olyan alkalmazást fejleszteni, amely nem az operációs rendszer környezetében, hanem internetes környezetben fut, így elkerülve a szükségét, hogy minden eszközre szükség legyen megírni az adott alkalmazást.

A weboldalak fejlődésük során kezdetben HTML jelölőnyelven íródtak, később azonban megjelentek a PHP, illetve a JavaScript neveken ismert nyelvek. Ezek  lehetőséget teremtettek a weboldalak valós programozására, a korábbi jelölőnyelves leírással szemben. Az említett nyelvek az évek alatt nagyon sokat fejlődtek funkcionalitásukban, mára viszont a fejlesztők körében gyakori kijelentés, hogy a "PHP halott nyelv", annak ellenére, hogy még mindig nagyon sok alkalmazás használja és a cégek aktívan keresnek programozókat a fejlesztésükre. Ennek az egyik oka a JavaScript elsöprő népszerűsége.

A  JavaScript (JS) egy ECMAScript specifikációt követő, magas szintű programozási nyelv. Nagyjából a weboldalak 97\%-a használja, így a legtöbb böngésző rendelkezik dedikált JavaScript motorral is. A JS-ben nincs natív támogatás az I/O-hoz, így azokat többnyire más környezettel biztosítják, például ilyen a Node.Js is. \cite{javascript}

A NodeJS egy aszinkron, eseményvezérelt futtatási környezet, amely alkalmas skálázható webes applikációk fejlesztésére. Csomagkezelője az úgynevezett Node Package Manager, röviden NPM. A csomagkezelő rendszerek közül a legtöbb csomagot az NPM tartalmazza.

Az NPM csomagok méretüket, funkcionalitásukat, felhasználásukat tekintve sokrétűek. Ilyen csomagként tartják nyilván a híresebb JavaScript keretrendszereket is, mint a Vue.JS, az Angular vagy a React. Ezek a keretrendszerek kényelmet biztosítanak a fejlesztő számára, illetve extra funkcionalitást nyújtanak a "vanilla" JavaScript-tel szemben, ehhez azonban sok esetben más NPM csomagokra is támaszkodnak működésük során, azaz függnek tőlük.

A készítendő program a fent említett keretrendszerek, illetve bármely NPM csomag függőségeinek elemzésére fog egy áttekinthető, interaktív és látványos megoldást adni. Az elemzé során a program képes lesz a csomagok függőségi gráfjainak ábrázolására és annak elemzésére, legyen szó egy csomagról vagy több csomag függőségeinek statisztikai elemzéséről.

A függőségek elemzésére azért lehet szükség, mert egy nagy mennyiségű függőséget felszórakoztató projekt fejlesztése során a felhasznált csomagoknál kialakulhat redundancia, azaz olyan eset, amikor két csomag nagyon hasonló problémára kíván megoldást, így célszerű lehet csak az egyiket felhasználni, így csökkentve a függőségeket. A program az elemzéseken túl, az említett esetlegesen felesleges függőségek kivonására próbál majd javaslatot adni.

A dolgozat következő fejezeteiben szó lesz a vizsgált technológia releváns területeinek az áttekintéséről, a készítendő program koncepciójának felvázolásáról, az implementáció megtervezéséről és megvalósításáról.
