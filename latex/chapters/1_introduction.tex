\Chapter{Bevezetés}

A szoftverfejlesztés egy folyamatosan változó és fejlődő terület. Az utóbbi időben rengeteg különböző célú és felhasználású technológia jött létre, azonban akadnak olyanok is, amelyek ugyanazt a célt valósítják meg, más megközelítéssel.

Kezdetben leginkább a személyi számítógépekre fejlesztettek programokat. A célhardverek halmaza azonban ma már sokkal nagyobb és színesebb. Az okoseszközök, televíziók, háztartási gépek, számítógépek, és egyéb eszközök gyakran más-más operációs rendszereket futtatnak, amelyekre más-más technológiákkal lehet fejleszteni alkalmazásokat.

Mivel a céleszközök nagyobb része támogatja valamely, vagy akár több böngésző használatát, így célszerű lehet olyan alkalmazást fejleszteni, amely nem az operációs rendszer környezetében, hanem internetes környezetben fut, így elkerülve a szükségét, hogy minden eszközre szükség legyen megírni az adott alkalmazást.

A weboldalak fejlődésük során kezdetben HTML jelölőnyelven íródtak, később azonban megjelentek a PHP, illetve a JavaScript neveken ismert nyelvek. Ezek  lehetőséget teremtettek a weboldalak valós programozására, a korábbi jelölőnyelvves leírással szemben. Az említett nyelvek az évek alatt nagyon sokat fejlődtek funkcionalitásukban, mára viszont a fejlesztők körében gyakori kijelentés, hogy a "PHP halott nyelv", annak ellenére, hogy még mindig nagyon sok alkalmazás használja és a cégek aktívan keresnek programozókat a fejlesztésükre. Ennek az egyik oka a JavaScript elsöprő népszerűsége.

A  JavaScript (JS) egy ECMAScript specifikációt követő, magas szintű programozási nyelv. Nagyjából a weboldalak 97\%-a használja, így a legtöbb böngésző rendelkezik dedikált JavaScript motorral is. A JS-ben nincs natív támogatás az I/O-hoz, így azokat többnyire más környezettel biztosítják, például ilyen a Node.Js is. \cite{javascript}

