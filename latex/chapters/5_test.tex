\Chapter{Tesztelés és Eredmények}

A program funckiói a fejlesztés során folyamatosan tesztelésre kerültek, alapvetően manuálisan. Az algoritmus, illetve a szemantikus verziózás tesztelése végül a statisztikai vizsgálat során automatikusan is megtörtént, bár nem ez volt annak a résznek a célja. Ennek következtében derült ki az algoritmusnál is tárgyalt hiba, miszerint nem számított bizonyos átlagtól eltér esetekre, amely a ciklus végtelen futását eredményezte.\\

Ebben a szekcióban a program felhasználása, illetve a használatával szerzett eredmények elemzése kerül ismertetésre.

\section{Használat}

\begin{figure}[!h]
	\centering
	\includegraphics[scale=0.2]{images/examiner.png}
	\caption{Egy Csomagos Mód}
	\label{fig:examiner}
\end{figure}

\begin{itemize}
	\item Első lépésként meg kell adni a csomag nevét és keresni a (Search) gombbal.
	\item Helyes csomagnév esetén feltöltődik a verziók listája, innen tetszőlegesen kell választani egyet.
	\item Opcionálisan megadható, hogy keresés mennyire legyen mély
	\item Végül a (Get dependencies) gomb megnyomásával lehet elindítani a folyamatot.
	\item Az elkészült ábra görgővel és egérgomb lenyomva tartásával interaktívan mozgatható, nagyítható. A Sidebar pedig lefelé görgethető a gráfelemzés információinak megtekintéséhez.
\end{itemize}

\pagebreak

\textbf{A két mód közötti váltást a (Switch) gomb segítségével lehet megtenni.}

\begin{figure}[!h]
	\centering
	\includegraphics[scale=0.2]{images/statistics.png}
	\caption{Több Csomagos Mód}
	\label{fig:examiner}
\end{figure}

\begin{itemize}
	\item Először a vizsgált csomagok mennyiségét szükséges megadni.
	\item A lefelé gördülő listából pedig a sorrendbe állítási elvet kell kiválasztani, alapértelmezett a leggyakrabban használt csomagok elve.
	\item A folyamatot a (Check Packages) gombra kattintással lehet elindítani.
\end{itemize}

\section{Eredmények}