\pagestyle{empty}

\noindent \textbf{\Large Használati útmutató}

\vskip 1cm

A szakdolgozat dokumentum és a program forráskódja a modern szoftver disztribúciós normákat követve megtalálható GitHub-on a következő linken:

\begin{verbatim}
	https://github.com/Petee99/package-examiner
\end{verbatim}

A számítógépen a program működéséhez szükséges a Node.JS megléte, amelyet az alábbi linken lehet elérni:

\begin{verbatim}
	https://nodejs.org/en/
\end{verbatim}

A "repository" letöltését követően a program mappájából indított terminálban szükséges a projekt függőségeinek telepítése:

\begin{verbatim}
	npm install
\end{verbatim}

A futtatásra két opció van:

\begin{itemize}
	\item Lokális fejlesztői szerver futtatása (\texttt{http://localhost:3000} lokális címen)
	\begin{verbatim}
		npm run dev
	\end{verbatim}
	\item Kiadásra alkalmas "build" készítés
	\begin{verbatim}
		npm run build
	\end{verbatim}
\end{itemize}

\noindent \textbf{Adathordozó}\\

A szakdolgozat pdf verziója, illetve a forráskód megtalálható az adathordozón. Továbbá megtalálható még rajta a Node.JS telepítője is, amennyiben szükséges.\\

Mind a telepítéshez, mind a függőségek telepítéséhez és a program megfelelő működéséhez is aktív internetkapcsolatra van szükség.\\

A telepítés menete a fent leírt módon történik.


